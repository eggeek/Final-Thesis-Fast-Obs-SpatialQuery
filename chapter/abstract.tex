Obstacle k-Nearest Neighbours problem is the k-Nearest Neighbour problem in a 
two-dimensional Euclidean plane with obstacles (\emph{OkNN}).
Existing and state of the art algorithms for OkNN are based on incremental 
visibility graphs and as such suffer from a well known disadvantage: costly 
and online visibility checking with quadratic worst-case running times.
In this research we develop a new OkNN algorithm which avoids these disadvantages
by representing the traversable space as a collection of convex polygons; i.e.
a Navigation Mesh. 
We then adapt a recent and optimal navigation mesh algorithm, \textit{Polyanya}, from the
single-source single-target setting to the the multi-target case. 
We also give two new and online heuristics for OkNN.
In a range of empirical comparisons, we show that our approach can be orders of magnitude faster than competing methods that rely on visibility graphs.

\textit{Keywords}: Obstacle Nearest Neighbor, kNN, Navigation Mesh, Spatial Search, Obstacle
Distance, Obstacle Navigation

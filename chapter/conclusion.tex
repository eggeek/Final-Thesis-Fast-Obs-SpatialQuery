\chapter{Conclusion}
\section{Contributions} \label{conc}
In this work we consider efficient algorithms for OkNN: the problem of finding $k$ nearest neighbours in a plane and in the presence of obstacles. We describe three new OkNN algorithma, all based on Polyanya~\cite{cuicompromise}, a recent and very fast algorithm for computing Euclidean-optimal shortest paths in the plane. The first variant involves brute force search (one query per target point). The second and third variants involves running Polyanya as a multi-target algorithm but with added heuristic guidance. We develop two new online and admissible heuristics for this purpose: the Interval Heuristic and the Target Heuristic.

We compare these variant algorithms against one another and against LVG~\cite{zhang2004spatial}, an influential and state of the art OkNN method based on incremental visibility graphs. The headline result from our experiment is that OkNN with Polyanya can be up to three orders of magnitude faster than LVG. Moreover, each of the three variants appears best suited to particular
OkNN settings: brute force search is highly effective when the number of candidates is small (independent of $k$); the Interval Heuristic works well when targets are many (again, independent of $k$); the Target Heuristic works well when targets are few and $k$ is also small.

\section{Future Research}
Due to their fast performance, we believe these algorithms can be used to speed up other types of spatial query which need to compute obstacle distance; e.g. as described in~\cite{gao2016reverse,gao2009continuous}.

Another interesting direction for future work is finding ways to further improve the Target Heuristic, which, in our experiments, accoutns for approximately $80\%$ of total running time. One possible approach involves combining the four \textit{R-tree} queries requried at present into one.

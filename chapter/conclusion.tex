\chapter{Conclusion and Future Work}\label{conclusionfuture}
\section{Research Contributions}\label{contribution}

In this work, we consider efficient algorithms for OkNN:
the problem of finding $k$ nearest neighbours in a plane and in the presence of obstacles.
We describe three new OkNN algorithms, all based on Polyanya~\cite{cuicompromise},
a recent and very fast algorithm for computing Euclidean-optimal shortest paths in the plane.
The first variant involves brute force search (one query per target point).
The second and third variants involve running Polyanya as a multi-target algorithm but with added heuristic guidance.
We develop two new online and admissible heuristics for this purpose: the Interval Heuristic and the Target Heuristic.

We compare these variant algorithms against one another and against LVG~\cite{zhang2004spatial},
an influential and state of the art OkNN method based on incremental visibility graphs.
The headline result from our experiment is that Polyanya OkNN can be up to
\textbf{three orders of magnitude} faster than LVG.

Moreover, each of the three variants appears best suited to particular OkNN settings:
\textbf{brute-force Polyanya} is highly effective when the number of candidates is small (independent of $k$);
the \textbf{Interval Heuristic} works well when targets are many (again, independent of $k$);
the \textbf{Target Heuristic} works well when targets are few and $k$ is also small.

The work has been accepted for publication in \textit{Proceedings of the 11th Annual Symposium on Combinatorial Search (SoCS’2018), colocated with
IJCAI/ECAI’2018, July 2018}.

\section{Future Works}\label{future}
We also consider future works in following directions.
Firstly, due to the high performance of proposed algorithms, we believe they can be used to
speed up other types of spatial queries which need to compute obstacle distance,
as described in section~\ref{lrquery}.

Secondly, we can improve heuristic function in \textit{target heuristic}.
In experiments, we notice that the \textit{target heuristic} cost $\approx 80\%$ of
the total running time on NN query, and one possible way is combining four queries into one.
Besides, in the literature review, we mainly focus on \textit{R-tree} and its variants which
have good scalability on dimensions, however, in our case, the dimension is always two (maybe
three in 3D extension). Thus, other high-performance and low-dimension data structures may helpful,
e.g. \textit{KD-tree}\cite{ooi1987spatial} and
\textit{Locality-sensitive hashing}\cite{slaney2008locality}.

Thirdly, \textit{brute-force Polyanya} can be improved by smart pruning. We notice that
the \textit{brute-force Polyanya} sometime outperforms other proposed algorithms in sparse scenarios,
thus, instead of considering all targets,
a pruning strategy like \textit{fast filter}\cite{xia2004fast} may make it work in general
scenario.
